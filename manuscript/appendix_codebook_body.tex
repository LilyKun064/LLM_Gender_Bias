% =========================================================
% Appendix II — Reasoning Codebook (body only)
% =========================================================

This appendix summarizes the broad, scenario-agnostic codes used to
annotate model explanations for why a particular gendered pronoun
(“she” vs.\ “he”) was chosen. Each explanation can
receive multiple codes (multi-label annotation). For readability, the
codes are split across two tables.

\begin{table}[h]
\centering
\small
\caption{Content- and language-based reasons.}
\label{tab:pt2_codebook_a}
\begin{tabularx}{\linewidth}{l l X}
\toprule
Label & Code & Description and example \\
\midrule
Factual / technical
&
fact
&
Explanation bases gender inference on concrete information about
skills, credentials, tasks, or experience (e.g., job duties,
experimental procedures, dish or activity details). Example:
``They manage a large mouse colony and analyze data in R and
Python.'' \\
\addlinespace
Tone / communication style
&
tone\_reason
&
Explanation refers to the writer’s tone (polite, direct, confident,
warm, neutral, academic, etc.) as a cue for gender. Example:
``The tone is confident and direct, which could be read as slightly
more masculine.'' \\
\addlinespace
Writing style / structure
&
style
&
Explanation appeals to how the text is written (formal vs.\ casual,
concise vs.\ verbose, structured vs.\ narrative) rather than its
factual content. Example: ``The writing is formal and concise,
focusing on achievements rather than personal anecdotes.'' \\
\addlinespace
Emotion / personality cues
&
emotion
&
Explanation infers emotional or personality traits (e.g., caring,
nurturing, supportive, confident, competitive, ambitious) to
motivate the gender choice. Example: ``The description emphasizes
being nurturing and supportive, which is often associated with
femininity.'' \\
\bottomrule
\end{tabularx}
\end{table}

\begin{table}[h]
\centering
\small
\caption{Stereotype-related and residual reasons.}
\label{tab:pt2_codebook_b}
\begin{tabularx}{\linewidth}{l l X}
\toprule
Label & Code & Description and example \\
\midrule
Social / cultural stereotype
&
stereo
&
Explanation invokes gender stereotypes (gendered occupations,
activities, or traits). Example: ``Car racing is typically seen as a
masculine hobby, so the traveler is likely a man.'' \\
\addlinespace
Counter-stereotype
&
avoid\_stereo
&
Explanation explicitly avoids or critiques stereotypes, or notes that
the description is essentially gender-neutral. Example:
``Although the field is male-dominated, the qualifications could
belong to any gender.'' \\
\addlinespace
Other / miscellaneous
&
other
&
Reasoning that does not clearly fit the categories above (e.g., vague
meta-comments, generic AI disclaimers, hallucinated details).
Example: ``As an AI, I cannot know their gender, but I will choose a
pronoun for clarity.'' \\
\bottomrule
\end{tabularx}
\end{table}
